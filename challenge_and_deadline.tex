\section{Challenges and Deadlines}\label{sec:challenges_and_deadlines}

A \emph{challenge} specifies a query that can be asked to a nonce.
The answer will be a \emph{deadline}.
Challenges will be applied to plots as well, by taking
the \emph{smallest} deadline for each nonce in the plot.
%
\begin{definition}[Challenge]\label{def:challenge}
  The set of \emph{challenges} is
  \[
  \Challenges=\left\{\langle\scoopnumber,\sigma\rangle\left|
  \begin{array}{l}
    0\le\scoopnumber<\numberofscoops\\
    \text{and $\sigma$ is a hash for $h_\generation$}
  \end{array}
  \right.\right\}.
  \]
  The $\sigma$ component of a challenge is said to be its \emph{generation signature}.
\end{definition}
%
In the following, generation signature will be used as a synonym of hash for $h_\generation$.
%
\begin{definition}[$\mathit{value}(\nonce,\challenge)$]\label{def:nonce_value}
  Let $\nonce\in\Nonces$ and $\challenge\in\Challenges$.
  The \emph{value $\mathit{value}(\nonce,\challenge)$ of $\nonce$ \wrt $\challenge$} is defined as
  \begin{multline*}
    \mathit{value}(\nonce,\challenge)\\
    =h_\deadline(\nonce.\scoops[\challenge.\scoopnumber]\append\challenge.\sigma).
  \end{multline*}
\end{definition}
%
\begin{definition}[Deadline]\label{def:deadline}
  The set of \emph{deadlines} is
  \[
  \Deadlines=\left\{
  \langle p,\pi,\mathit{value},\challenge\rangle
  \left|\begin{array}{l}
  \pi\in\Prologs,\ p\in\mathbb{N}\\
  \mathit{value}\text{ is a hash for }h_\deadline\\
  \text{and }\challenge\in\Challenges
  \end{array}
  \right.
  \right\}
  \]
  Deadlines are totally ordered by increasing value.
\end{definition}
%
Intuitively, the value of a deadline expresses how many milliseconds
must be waited until the deadline expires and a new block can be mined.
However, if the mining power of the network increases, the minimal value of the deadlines
generated by the network
tends to decrease, and vice versa. That is, the block creation rate would not be
fixed to $\beat$ (Tab.~\ref{tab:notations}), on average.
This explains why one needs to modulate the value of the deadlines \wrt a number
called \emph{acceleration}\footnote{In~\cite{SignumPlotting} the term
\emph{base target} is used for it, but we think that \emph{acceleration} is clearer.},
which is the inverse of Bitcoin's difficulty.
%
\begin{definition}[Deadline's waiting time]\label{def:deadline_waiting_time}
  Given $\delta\in\Deadlines$ and an \emph{acceleration}
  $\alpha\in\mathbb{N}$ such that $\alpha>0$, the
  \emph{waiting time} for $\delta$ \wrt $\alpha$ is
  %
  \[
  \waitingtime(\delta,\alpha)
  =\betonat\left(\text{first $8$ bytes of}\left(\nattobe\left(\frac{\betonat(\delta.\mathit{value})}{\alpha}\right)\right)\right).
  \]
\end{definition}
%
The following algorithm shows how a nonce answers a challenge with a deadline.
%
\begin{alg}[$\delta(\nonce,\pi,\challenge)$]\label{alg:deadline_from_nonce}
  Given $\nonce\in\Nonces$, $\pi\in\Prologs$ and $\challenge\in\Challenges$, the
  \emph{deadline computed from $\nonce$ for $\pi$ and $\challenge$} is
  \[
  \delta(\nonce,\pi,\challenge)=\langle\nonce.p,\pi,\mathit{value}(\nonce,\challenge),\challenge\rangle.
  \]
\end{alg}
%
Alg.~\ref{alg:deadline_from_nonce} is extended to plots. Remember that plots are non-empty
(Def.~\ref{def:plot}) and that deadlines are ordered by their value.
%
\begin{alg}[$\deadline(\plot,\challenge)$]\label{alg:deadline_from_plot}
  Given $\plot\in\Plots$ and $\challenge\in\Challenges$, the \emph{deadline computed
  from $\plot$ for $\challenge$} is
  \[
  \delta(\plot,\challenge)=\min\limits_{\nonce\in\plot.\nonces}\delta(\nonce,\plot.\pi,\challenge).
  \]
\end{alg}
%
A deadline is valid when the nonce built from its progressive and prolog
has the same value as the deadline \wrt its challenge.
%
\begin{definition}[Deadline's validity]\label{def:deadline_validity}
  Let $\delta\in\Deadlines$. We say that $\delta$ is \emph{valid} if and only if
  \[
  \delta.\mathit{value}=\mathit{value}(\nonce(\delta.p,\delta.\pi),\delta.\challenge).
  \]
\end{definition}
%
The consensus rules of the blockchain, later, will check that the deadlines are
actually valid (Def.~\ref{def:blockchain}). Therefore, one could be tempted to
remove the $\mathit{value}$ field from deadlines (Def.~\ref{def:deadline}) and compute
their value, when needed, by Def.~\ref{def:deadline_validity}. However, this would change the
computational complexity of the algorithms, since the computation of
$\nonce(\delta.p,\delta.\pi)$ is relatively expensive (Def.~\ref{alg:nonce_construction}).
Therefore, the original informal description~\cite{SignumPlotting} puts a value in the deadline
and we follow this approach as well.
