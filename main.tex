\documentclass[orivec]{llncs}
\usepackage{amsmath}
\usepackage{amssymb}
\usepackage{wrapfig}
\usepackage{stmaryrd}
\usepackage{graphicx}
\usepackage[T1]{fontenc}
\usepackage{array}
\usepackage{listings, framed}
\lstset{
  language=Java,
  showstringspaces=false,
  columns=flexible,
  basicstyle={\ttfamily},
  frame=l,
  numbers=left,
  numberstyle={\ttfamily},
%  breaklines=true,
  breakatwhitespace=true,
  tabsize=3,
  escapechar=|
}
%\usepackage{algorithm, algorithmic}
\usepackage{url}
\usepackage{relsize}
%\usepackage{xcolor}
%\usepackage{multirow}

%\newcommand{\PreserveBackslash}[1]{\let\temp=\\#1\let\\=\temp}
%\newcolumntype{C}[1]{>{\PreserveBackslash\centering}p{#1}}
%\newcolumntype{R}[1]{>{\PreserveBackslash\raggedleft}p{#1}}
%\newcolumntype{L}[1]{>{\PreserveBackslash\raggedright}p{#1}}

%\newif\iflongversion
%\longversionfalse               % for conference version
%\longversiontrue                % for technical report

% \|name| or \mathid{name} denotes identifiers and slots in formulas
\def\|#1|{\mathid{#1}}
\newcommand{\mathid}[1]{\ensuremath{\mathit{#1}}}
% \<name> or \codeid{name} denotes computer code identifiers
\def\codesize{\small}
\def\<#1>{\codeid{#1}}
\newcommand{\codeid}[1]{\ifmmode{\mbox{\codesize\ttfamily{#1}}}\else{\codesize\ttfamily #1}\fi}

\newcommand{\todo}[1]{{\color{red}\bfseries [[#1]]}}

\DeclareMathOperator*{\append}{\bowtie}

\newcommand{\fs}[1]{\todo{FS: #1}}
\newcommand*\xor{\oplus}

% Reduce padding for \boxed
%\setlength{\fboxsep}{.5\fboxsep}

\newtheorem{alg}{Algorithm}

\newcommand{\wrt}{\textit{wrt.\ }}
\newcommand{\ie}{, \textit{ie.}, }

\newcommand{\numberofscoops}{{\#\mathit{scoops}}}
\newcommand{\noncesize}{{\mathit{nonce}_{\mathit{size}}}}
\newcommand{\byte}{{\mathit{byte}}}
\newcommand{\nonce}{{\mathit{nonce}}}
\newcommand{\nonces}{{\mathit{nonces}}}
\newcommand{\scoops}{{\mathit{scoops}}}
\newcommand{\plot}{{\mathit{plot}}}
\newcommand{\seed}{{\mathit{seed}}}
\newcommand{\deadline}{{\mathit{deadline}}}
\newcommand{\generation}{{\mathit{generation}}}
\newcommand{\size}{{\mathit{size}}}
\newcommand{\scoopnumber}{{\mathit{scoopNumber}}}
\newcommand{\data}{{\mathit{data}}}
\newcommand{\challenge}{{\mathit{challenge}}}
\newcommand{\mynext}{{\mathit{next}}}
\newcommand{\nextchallenge}{\challenge_\mynext}
\newcommand{\initialchallenge}{{\mathit{initialChallenge}}}
\newcommand{\height}{{\mathit{height}}}
\newcommand{\trunk}{{\mathit{trunk}}}
\newcommand{\block}{{\mathit{block}}}
\newcommand{\oblivion}{{\mathit{oblivion}}}
\newcommand{\now}{{\mathit{now}}}
\newcommand{\beat}{{\mathit{beat}}}
\newcommand{\weightedbeat}{\mathit{weightedBeat}}
\newcommand{\nextweightedbeat}{\weightedbeat_\mynext}
\newcommand{\power}{{\mathit{power}}}
\newcommand{\previousblockhash}{{\mathit{previousBlockHash}}}
\newcommand{\waitingtime}{{\mathit{waitingTime}}}
\newcommand{\nextwaitingtime}{\tau_\mynext}
\newcommand{\nextpower}{\power_\mynext}
\newcommand{\nextacceleration}{\alpha_\mynext}
\newcommand{\genesis}{{\mathit{genesis}}}
\newcommand{\consensus}{{\mathit{consensus}}}
\newcommand{\nattobe}{{\mathit{nat2be}}}
\newcommand{\betonat}{{\mathit{be2nat}}}
\newcommand{\finalhash}{{h_{\mathit{final}}}}

\newcommand{\Prologs}{{\mathsf{Prologs}}}
\newcommand{\Plots}{{\mathsf{Plots}}}
\newcommand{\Challenges}{{\mathsf{Challenges}}}
\newcommand{\Scoops}{{\mathsf{Scoops}}}
\newcommand{\Nonces}{{\mathsf{Nonces}}}
\newcommand{\Trunks}{{\mathsf{Trunks}}}
\newcommand{\Blocks}{{\mathsf{Blocks}}}
\newcommand{\GenesisBlocks}{{\mathsf{GenesisBlocks}}}
\newcommand{\NonGenesisBlocks}{{\mathsf{NonGenesisBlocks}}}
\newcommand{\Deadlines}{{\mathsf{Deadlines}}}

%
\begin{document}
%
\begin{frontmatter}
  \title{A Formalization of Signum's Proof of Space}
\author{Fausto Spoto}
\institute{Dipartimento di Informatica, Universit\`a di Verona, Verona, Italy\\
  \email{fausto.spoto@univr.it}}
%
\maketitle
%
\begin{abstract}
  Very interesting text.
\end{abstract}
%
\end{frontmatter}

\section{Introduction}\label{sec:introduction}

A blockchain is a data structure where \emph{transactions} are kept inside blocks.
Blocks form a chain (a list), where each block $b$ \emph{points} to its previous block $p$
by referring, inside $b$, to the hash of $p$. Blocks must satisfy some consistency
rules, called \emph{consensus} rules; for instance, if $b$ refers to a previous bock hash
$h$ then a block $p$ having that hash must really exist in the chain; moreover, the timestamp
of $b$ must be larger than that of $p$; the size of $b$ must be smaller than a given threshold and so on.
The exact nature of the transactions is not relevant in this paper. In general, they are
requests to update the state of a global abstract machine; this state might be a ledger of
payments (as in the case of Bitcoin~\cite{Nakamoto08,Antonopoulos17}) or a sort of global RAM where data
structures can be allocated and subsequently modified (as in the case of Ethereum~\cite{AntonopoulosW18}).

The use of hashes as machine-independent
pointers allows blockchains to be implemented in a distributed way, in a network of peers.
Distributions is a desirable property because it entails that data is safely duplicated
in each peer and that there is no special peer that determines the history of the transactions.
However, each peer is free to expand the blockchain with new blocks, independently from the other
peers, so that, in general, there
are more blocks $b$ that refer to the same previous block $p$ and the blockchain is a tree rather than
a list of blocks. In order to make a single chain emerge as the \emph{best} chain, a notion of
chain quality is used: peers have incentives to append blocks to the chain with the highest quality.
This entails that a peer might replace its current best chain with another, even better chain,
if it receives the latter from other peers. This event is a \emph{history change}.

The above description of a distributed blockchain is still missing a key ingredient. Namely,
as presented above, each peer is free to generate new blocks at maximal speed, flooding the network
with new blocks, making difficult the emergence of a best chain and inducing frequent history changes.
This is not just an efficiency problem but also a security problem: history changes allow
\emph{double spending}, when the same money is moved in the ledger twice, once in the previous history
and once in the updated history. The actual genious of Nakamoto~\cite{Nakamoto08} has been to (largely)
solve these issues in a very elegant way, by exploiting an idea previously developed for combatting
email spam~\cite{DworkN92}. Namely, he added a consensus rule requiring that the (binary)
hash of each block $b$ must start with at least $\delta$ zeros, and connected the quality of a chain
to this hash. This means that the creation of a new block requires to rotate among
many possible values for a block field, called \emph{nonce}, until the hash of the block satisfies the
added consensus rule. This makes the creation of new blocks hard
(for larger $\delta$), makes it impossible to create blocks at arbitrary speed and creates a heavy
incentive to expanding the best chain, rather than creating alternative histories, since otherwise a peer
risks spending work (concretely, electricity) for the creation of blocks that will be discarded by the other peers.
The value $\delta$ is called \emph{difficulty} and is not a constant: it changes in accordance with
the current, total computational power of the network, in order to keep the block creation rate at a predetermined value.
The process of finding a good nonce, that induces an
acceptable block hash, is the \emph{proof of work} algorithm: it is a brute force algorithm, because of the non-correlation
property of hash functions. A peer that performs the proof of work algorithm is said to \emph{mine}
a new block and is consequently a block \emph{miner}. Miners get remunerated for their work whenever
they mine a new block before all other miners. Theoretically, everybody can install a miner peer,
anonymously, which makes the idea of Bitcoin very democratic.

The proof of work secures a blockchain network, reducing the risk of double spending,
but comes at the price of energy consumption: the electricity used by the Bitcoin network is said
to be comparable to that of a medium-sized country; moreover, mining is not egalitarian, because
it is worthwhile only in countries where electricity is cheap; furthermore, the proof of work algorithm
is more efficient in dedicated, relatively expensive hardware (such as ASICs),
which deviates much from the idea of a democratically open network.

In order to overcome the issue with energy consumption, the recent trend in blockchain is to
replace the proof of work with a proof of stake. This comes in many different flavors, but
the shared idea is that mining is limited to a (static or dynamic, exclusive or delegatable)
set of peers, that commit
some cryptocurrency (a \emph{stake}) to gain the right of mining
in turn, or according to some alternation protocol. In general, these can be seen
as Byzantine consensus algorithms, as the one pioneering by
Tendermint (\url{https://github.com/tendermint/tendermint/wiki}).
Proof of stake is often criticized for being more centralized and less democratic than proof of work
(\emph{rich becomes richer}).
Moreover, it suffers from what we call the \emph{start-up issue}: as long as the cryptocurrency
of a newborn blockchain has still no value, it is difficult to convince miners to work and
be updated, since there is no incentive in doing so, initially. Starting and maintaining
a newborn blockchain becomes a difficult social and organizational problem. Finally, peers of
a proof of stake blockchain get punished (\emph{slashed}) if they are misbehaving or offline.
This is problematic if, for instance, a peer is offline but has no fault for that: it might be
because of a network connectivity issue or a black-out.

Among the alternatives to proof of work and proof of stake, we focus here on proof of space.
In a proof of space blockchain, peers have the right to mine new blocks (and be renumerated for that) if they dedicate
a large chunk of disk memory for mining. The energy consumption of proof of space is almost zero
and no dedicated hardware can be used for mining, currently. Therefore, mining becomes cheap and
more democratic than with a proof of work.
The theoretical background of proof of space has been developed, independently,
in the two seminal papers~\cite{AtenieseBFG14} and~\cite{DziembowskiFKP15},
that feature similarities but also significant differences. Both are based
on directed acyclic graphs (DAGs) of high pebbling complexity.
Pebbling, here, is a directed decoration of the nodes of the DAG with hashes, as in
a Merkle tree.
A prover must keep such a (big) DAG and its pebbling on disk, in order to answer, efficiently,
\emph{challenges} proposed by a verifier, with compact proofs that should convince the verifier that
the prover is actually keeping the DAG on disk. These proofs are used for mining new blocks,
instead of the nonce used in the proof of work. A notion of quality is defined for
the proofs, in such a way that the probability of deriving a proof of high quality increases
with the size of the DAG, which is an incentive to dedicating more space for mining.
While~\cite{DziembowskiFKP15} requires space to remain allocated between challenges,
and is consequently called a proof of \emph{persistent} space, \cite{AtenieseBFG14} requires
to allocate space only when anwering challenges and is consequently called
a proof of \emph{transient} space (or a proof of secure erasure, as~\cite{DziembowskiFKP15} calls it).
Both solutions have an initialization phase when the verifier performs a deeper challenge
of the prover and stores the resulting proof in blockchain.

%Cite~\cite{DziembowskiFKP15}. This seems to be the first description of proofs of space.
%Their algorithm is based on graph pebbling, where a vertex can be pebbled only if the
%its in-going vertices have been pebbled as well. This way they prove a lower bound on the
%complexity of their algorithm. They prove that that lower bound is valid also if a prover
%wants to use its CPU. They prove that the size of the space used by their algorithm
%is a lower bound to the execution cost of the algorithm if no space is reserved.
%Therefore, using proofs of work in a network of proofs of space nodes would
%be computationally too expensive.
%This is sometimes called proof of persistent space.
%There is an initialization protocol for each new prover, that is missing in Burstcoin.

%Cite~\cite{AtenieseBFG14}. Based on DAGs with high pebbling complexity. There are clear similarities
%with~\cite{DziembowskiFKP15}. They actually cite and compare with each other.
%According to~\cite{DziembowskiFKP15}, this article defines a proof of secure erasure,
%that however they call a proof of space. According to~\cite{DziembowskiFKP15},
%their proof of secure erasure
%implies a proof of space but not the other way round. There does not seem to exist
%any implementation. The issue with the size of the proofs would be identical to that
%of~\cite{DziembowskiFKP15}, because of the use of pebbling graphs.
%This is sometimes called proof of transient space: the puzzle function requires lot of
%memory space to compute, but after computation that space con be freed.

Cite~\cite{TangZDWLG0L19}. They tackle the problem of miners using the same stored data
for many chains. This is problematic for newborn chains, since they might get an attack
from miners (or coalition of miners) that already use a very big data file for mining other,
mature chains (\emph{newborn attack}). They present a solution over the SpaceMint protocol.
The idea is that the disk space of a miner can be split for mining on many chains
simultaneously, but there is an incentive at allocating a space, for each chain, proportional
to the market value of the chain.

Cite~\cite{RenD16}. It uses stacked expander graphs, to get simpler, more efficient and
provably space-hard solutions, than~\cite{AtenieseBFG14} and~\cite{DziembowskiFKP15}.
It works for both proof of transient space and proof of persistent space.
It performs a nice comparison of previous proofs of space
and related techniques (memory-hard functions, proof of secure erasure, provable data possession,
proof of retrievability).

Proof of retrievability~\cite{JuelsK07}: a large file
is sent from the verifier to the prover and the verifier
check, repeatedly, if the prover keeps that file in storage. It requires to transfer the file
at the beginning, for each new prover (miner).

Cite~\cite{ParkKFGAP18}. SpaceMint, previously Spacecoin. Consideration: PoW requires power
to be allocated if mining is worthwhile. PoS allows one to allocate unused space even if its
cost is higher than mining, since in any case it would remain unused. More egalitarian:
general-purpose hardware instead of ASIC. The use of a key for the miners makes it
impossible to build mining pools, which is said to be good, citing~\cite{MillerKKS15}.
They say PoS is more difficult to adapt to blockchain because the protocol is a bit
more complicated than PoW. It lists some problems of PoS: mining multiple chains simultaneously
(since they are cheap), creating more blocks with the same proof and then choose the most
favorable (block grinding). Nothing-at-stake problems. Quality-function to determine the winner, proportional
to the allocated space. How to fight block grinding: make the proof unique, based
on who won the previous round, use
two chains, for proofs and for transactions, the proofs depends on previous proofs only.
How to fight mining on multiple chains: previous blocks affect future blocks only in a limited way.
They provide a game-theoretic model showing that the system is a Nash equilibrium.
It cites proof of storage/retrievability: the verifier must send and keep a big file.
Some link with Permacoin, which is however still a PoW system with ethical data.
It cites Burstcoin (now Signum) and its time/memory tradeoff:
``a miner doing a little extra computation can mine at the same
rate as an honest miner, while using just a small fraction (e.g., 10\%) of the space.''
It talks about a problem with miners hashing 8 million blocks, that does not seem to exist
anymore, but better check what they mean.
It cites the Chia Network (proof of space and time), \url{https://www.chia.net/}.
It calls it proof of sequential work on top of proof of space.
It says that it is based on completely different theoretical work, that is~\cite{AbusalahACKPR17}.
Consideration: their mining uses special protocol transactions
(payments, space commitments, penalties) while Mokamint is completey transaction agnostic.
Arrival of new miners and penalties for miners are kept in blockchain!
To avoid mining for different chains, the next challenge is derived from the hash of a block
(from the proof chain) deep in the past. If two children blocks are created by the same miner,
a penalty transaction is generated. The transaction includes the two blocks (it is huge!)
that are consequently signed, which guarantees that it can be verified by nodes that might
only have one history in the database.
The same challenge is used for a few consecutive blocks, to fight challenge-grinding attacks.
The size of their proofs (node pebbling) reaches 3 megabytes. This prove is stored in blockchain
for the initialization of each new miner (which might be expensive) and cheaper proofs
(100K, thanks to some ``likely sound'' optimations) are reported in each mined block.
Mokamint's deadline have constant (small size).
Code of Spacemint: \url{https://github.com/kwonalbert/spacemint}. Just a very limited prototype of a
proofs of space algorithm. Not maintained in the last nine years.

Cite~\cite{AbusalahACKPR17}. It is the theoretical base of the Chia network.
They assert that pebbling-based approaches have two main issues: the size of the proof to include in each block
(megabytes) and the initialization phase for each prover joining a verifier. That is, crypto must be spent even before
starting mining (the prover is the miner in this context) while Bitcoin allows one
to start mining and collect crypto on the way.
They solve these issues (the same is solved in Burstcoin as well).
They propose a proof of space and time based on challenges about the inversion of a random function,
overcoming the well-known issue of time-memory trade-offs.
As concisely stated in~\cite{ParkKFGAP18}, the better the quality of the proof of space, the faster the block can
be \emph{finalized} by a proof of sequential work, and this
proof tuple then can be used to create a block.

Cite~\cite{Reyzin23}. It tackles the question: how much does it cost to store only a part of the file?
When storing less than all of the file, it should be difficult for the prover to recover the
missing portions of the file when answering queries from the verifier. They define some thresholds
on the portion kept for the file and on the consequent complexity degradation. Ideally, such thresholds
should be close to 0, meaning that almost all file must be kept in memory for having no complexity
explosion. They show that existing solutions have bad constants or can be considered as impractical.
They say that the initialization protocol of the DAGs prevents most cheating (incomplete calculations).
The missing pebbles (red nodes) must be calculated later during the execution protocol.
They provide lower bounds on the cosntants of the initialization protocol so that
the resulting partial pebbling has thresholds close to 0.
This article tackles a problem that actually affected Burstcoin (its time/memory trade-offs,
theoretically now solved in Burstcoin). However, proofs in this article donot apply to Burstcoin
because it is not based on pebbling graphs.

Cite~\cite{DworkN92}. The origin of proof of work. Mail senders must compute some
work to have their email accepted by the recipient. This work includes
the address of the recipeint and the date of sending, in order to avoid
work recycling. Typically, it consists in adding extra data at the end of the email
to that its hash is smaller than a predefined constant.

Burstcoin, now Signum: \url{https://wiki.signum.network/}. Rebranded from Burstcoin.
They call it proof of capacity but it's just proof of space.
There is a very raw description of the mining algorithm:
\url{https://wiki.signum.network/signum-plotting-technical-information/index.htm}.
It allows smart contracts in ``Java'', in a language called SmartJ. Examples
can be found at \url{https://github.com/signum-network/signum-smartj}.
While Hotmoka abstracts away all blockchain details, SmartJ requires programmers to
manually deal with them.

\section{Related Work}\label{sec:related_work}

Proof of work was originally meant as protection against email spam~\cite{DworkN92}:
senders must perform some work to have their emails accepted by recipients.
%The input includes
%the address of the recipient and the date of sending, in order to ban
%recycling of work.
%The algorithm adds extra data at the end of the email,
%which corresponds to the nonce used in Bitcoin.
Ethereum started with proof of work~\cite{AntonopoulosW18} and later
moved to proof of stake. The latter can be seen as a
Byzantine consensus algorithm, as pioneered by Tendermint~\cite{Kwon14}.
Most current blockchains use some form of proof of stake nowadays.

The theoretical background of proof of space was independently developed
in two seminal papers~\cite{AtenieseBFG14,DziembowskiFKP15}.
%They feature both similarities and significant differences.
Both are based on directed acyclic graphs (DAGs) of high pebbling complexity.
Pebbling, here, is a directed hash decoration of the nodes of the DAG, as in a Merkle tree.
A prover must hold such a (big) DAG and its pebbling on disk, in order to answer, efficiently,
challenges with proofs that should convince a verifier that
the prover is actually holding data on disk.
While~\cite{DziembowskiFKP15} requires space to remain allocated between challenges
(proof of \emph{persistent} space), \cite{AtenieseBFG14}~requires one
to allocate space only when answering challenges
(proof of \emph{transient} space or \emph{proof of secure erasure}, as~\cite{DziembowskiFKP15} calls it).
Both solutions have an initialization phase, when the verifier performs a deeper challenge
of the prover and stores the resulting (big) proof in blockchain, followed by an execution phase,
when the verifier challenges the prover for each new block. Also~\cite{RenD16} uses
pebbling for stacked expander graphs, to get simpler, more efficient and
provably space-hard solutions.
It works for both proof of transient space and proof of persistent space.
It includes a nice review of previous proof of space
and related techniques: memory-hard functions, proof of secure erasure, provable data possession,
proof of retrievability.

Time/memory tradeoffs are studied in~\cite{Reyzin23}. They occur if
a prover can store only a part of the data on disk, with a less than proportional
degradation of its mining capabilities.
A good proof of space algorithm should make it difficult for a prover to recover the
missing part, when answering challenges.
In the context of graph pebbling, the initialization phase prevents most cheating
(that is, keeping incomplete data on disk). In~\cite{Reyzin23}, the size
of the portion of the file not kept on disk is related to the consequent time
complexity degradation for computing the missing part.
Ideally, the full file and pebbling must be kept in memory
for having no time complexity explosion, but they show
that this is not the case for existing solutions and
provide sufficient conditions for the initialization phase, that guarantee the ideal result.

The use of graph pebbling seems to dominate the literature on proof of space.
Instead, \cite{AbusalahACKPR17}~proposes an alternative technology that is the
theoretical background of the Chia network~\cite{CohenP19,Chia}:
a proof of sequential work on top of a proof of space, based on challenges
about the inversion of a random function, for which time/memory tradeoffs have been solved.
This is not, however, a pure proof of space.
Another alternative is the proof of retrievability
in~\cite{JuelsK07}: the verifier sends, initially,
a large file to the prover (miner) and later challenges, repeatedly, the prover to see
if it still keeps the file in storage. Its apparent simplicity
is jeopardized by the fact that big files must be shared.
In a blockchain network, they must be shared among \emph{all} (present and future)
peers, for \emph{all} (past, present and future) miners, or otherwise peers
could not verify the blocks.

We are aware of only one implementation of graph-pebbling proof of space:
SpaceMint~\cite{ParkKFGAP18}, previously Spacecoin~\cite{ParkPAFG15}.
It is not a blockchain implementation,
but a prototype of the proof of space protocol of~\cite{DziembowskiFKP15}.
Its code~\cite{SpaceMintCode} has not been maintained in the last nine years.
Nevertheless, \cite{ParkKFGAP18}~exposes interesting problems (and solutions),
related to the actual game theory and implementation of proof of space.
For instance, it uses the public key of the verifier as an input parameter
for pebbling, to discourage the creation of mining pools, often seen
negatively~\cite{MillerKKS15}.
Furthermore, it provides solutions for nothing-at-stake problems. Against block grinding,
it makes challenges independent from the transactions in the blocks, by splitting the
blockchain in a proofs blockchain and in a transactions blockchain:
only the first is used for mining,
and the two are connected with the signature of the miner.
Against challenge grinding, it lets past blocks influence the quality of short sequences
of future blocks only. A similar but more drastic solution, in~\cite{CohenP19}, is to
use the same challenge for several consecutive blocks,
since it is unlikely that a challenge will be good for many consecutive blocks.
Against mining on multiple chains, \cite{ParkKFGAP18} proposes
a technique that spots such behavior and
imposes a penalty transaction against the culprit.
%The latter includes pairs of blocks, as evidence, which is
%problematic since transactions are, in turn, included in blocks and this makes it impossible to define
%a maximal block size, a basic security requirement of every blockchain.
%We have discussed this with K.\ Pietrzak who suggested that this could be actually optimized
%by reporting signed hashes only.

Actual experiments are reported in~\cite{ParkKFGAP18}
that provide an estimation of the size of Spacemint's proofs:
in the initialization phase, they are between two and three megabytes;
in the execution phase, they can be optimized to around 100 kilobytes.
Such proofs must be persistently stored in
blockchain (for each new miner, in the first case, and for each new block, in the second).
This makes the size of the blockchain much larger than in Bitcoin, and requires
that miners hold cryptocurrency to store proofs even before starting mining.

Signum~\cite{Signum}, previously Burstcoin, has been launched in 2014, with possibly
the first ever language for smart contracts, and is still active.
Appendix B of~\cite{ParkKFGAP18} reports a formalization of an old version
of Alg.~\ref{alg:nonce_construction}.

Newborn attacks are considered in~\cite{TangZDWLG0L19}. Their solution is
to split the space for mining on many chains, with an
incentive to allocate, for each chain, a space proportional
to the market value of the chain.

\section{Nonces and Plots}\label{sec:nonces_and_plots}

This section formalizes the notion of plot, a collection of nonces, together with their
algorithmic construction. Signum's algorithm requires miners to hold one or more plots
in their memory. Their initialization is performed only ones and offline, hence it is
not part of the mining protocol.

Tab.~\ref{tab:notations} collects some notations,
introduced here and used throughout the paper.
It also reports the specific choices
made in~\cite{SignumPlotting} for such notations, but this paper remains parametric \wrt them.
Note that we distinguish three hashing functions in our formalizations, although they
might actually coincide. We prefer to remain as generic as possible.

\begin{table}[t]
\begin{center}
\begin{tabular}{|c|c|c|}
  \hline
  \textbf{notation} & \textbf{meaning} & \textbf{in~\cite{SignumPlotting}}\\\hline\hline

  $\append$ & concatenation on sequences & \\\hline
  $\numberofscoops$ & number of scoops contained in a nonce & $4096$\\\hline

  $h_\deadline$ & hashing function for computing nonces, plots and deadlines & shabal256\\\hline

  $h_\generation$ & hashing function for computing the generations of challenges & shabal256\\\hline

  $h_\block$ & hashing function for computing the hash of the blocks & \\\hline

  $\kappa$ & threshold to the number of bytes fed to $h_\deadline$ in Alg.~\ref{alg:nonce_construction} & $4096$\\\hline

  $\beat$ & target block creation time interval (ms) & $240000$ \\\hline

  $\sigma_\genesis$ & generation signature for the genesis block & \\\hline

  $\tau_\now$ & current time (ms from Unix epoch) & \\\hline

  $\oblivion$ & acceleration reaction to changes of mining power (from $0$ and $1$) & \\\hline
\end{tabular}
\end{center}
\caption{Notations and contextual information used in our formalization and their specific instantiations
  used in~\cite{SignumPlotting}, when available.}
\label{tab:notations}
\end{table}

The following definitions are used to deal with bytes and hashing.
%
\begin{definition}[Concatenation operator $\bowtie$]
  Sequences (for instance, of bytes) are concatenated by $\bowtie$. The same $\bowtie$
  is used to concatenate a sequence to a element or an element to a sequence.
\end{definition}
%
\begin{definition}[$\nattobe$ and $\betonat$]
  The operators $\nattobe$ and $\betonat$ transform natural numbers
  into their big-endian representation, and vice versa.
\end{definition}
%
\begin{definition}[Hashing function]
  A \emph{hashing function} $h$ of $\size>0$
  is a total map $h:\mathit{byte}^*\to\mathit{byte}^\size$, where
  $\mathit{byte}^*$ is a sequence of bytes, of arbitrary length,
  and $\mathit{byte}^\size$ is a sequence of $\size$ bytes, called a \emph{hash} for $h$.
  If $h$ is a hashing function, then $\size(h)$ is its size.
\end{definition}

A \emph{scoop} is a pair of hashes.
A \emph{nonce} is a non-negative \emph{progressive number} $p$, and
a list of $\numberofscoops>0$ scoops or, equivalenty,
a list of $2\cdot\numberofscoops$ hashes for $h_\deadline$.
Their definitions are parametric \wrt a hashing
function $h_\deadline$ used for their creation.
%
\begin{definition}[Scoop, Nonce]
  The sets of \emph{scoops} and \emph{nonces} are
  \[
  \Scoops=\left\{\langle h_1,h_2\rangle\left|\begin{array}{l}
  h_1,h_2\text{ are hashes for }h_\deadline
  \end{array}\right.\right\},
  \]
  \[
  \Nonces=\left\{\langle p,\scoops\rangle\left|\begin{array}{l}
  p\in\mathbb{N}\text{ and }\scoops\in\Scoops^\numberofscoops
  \end{array}\right.\right\}.
  \]
\end{definition}
%
In the above definition, angular brackets stand for tuples. In this specific
case, they stand for pairs. When definitions are given in terms of tuples, as above
for $\Scoops$ and $\Nonces$, we silently assume that there are selection functions
for the elements of the tuple. For instance, if $\nonce\in\Nonces$, then $\nonce.p$
and $\nonce.\scoops$ are the fields of the pair $\nonce$.

A \emph{prolog} is the identifier of the creator of nonces and plots (for instance, its public key).
For now, it is just a sequence of bytes. Sec.~\ref{sec:attacks_and_protections}
will give structure to prologs and see how they can be useful.
%
\begin{definition}[Prolog]\label{def:prolog}
  The set of \emph{prologs} is
  \[
  \Prologs=\{\pi\mid\pi\in\mathit{bytes}^*\}.
  \]
\end{definition}
%
The following algorithm constructs a nonce, given its progressive number and a prolog.
It uses a constant $\kappa>0$ (Tab.~\ref{tab:notations})
to limit its computational cost, hence avoiding to hash very large chunks of data.
%
\begin{alg}[$\nonce(p,\pi)$]\label{alg:nonce_construction}
  Given $p\in\mathbb{N}$ and $\pi\in\Prologs$, let
  \[
  \nonce(p,\pi)=\langle p,\langle h_0,h_1\rangle\append
  \cdots\append\langle h_{2\cdot\numberofscoops-2},h_{2\cdot\numberofscoops-1}\rangle\rangle\in\Nonces,
  \]
  where the hashes $h_0,\ldots,h_{2\cdot\numberofscoops-1}$ are constructed as
  follows\footnote{Step.~\ref{step:nonce_construction:swap} has been added
  after the formalization of this algorithm given in~\cite{ParkPAFG15} and
  in response to their criticisms. See Sec.~\ref{sec:attacks_and_protections} for a discussion.}.
  %
  \begin{enumerate}
  \item Let $\seed=\pi\append\nattobe(p)$.
  \item\label{step:nonce_construction:first_hash}
    For each $i$ from $2\cdot\numberofscoops-1$ to $0$,
    let\footnote{In~\cite{SignumPlotting}, it is said to
    take the \emph{last} $\kappa$ bytes, but an inspection of their code
    shows that they actually take the \emph{first} $\kappa$ bytes. They
    probably use \emph{last} here in the sense of \emph{more recently computed}.
    This threshold $\kappa$ seems to have been added after the formalization of
    this algorithm given in~\cite{ParkPAFG15}, where it is not present.}
    \[
    h_i=h_\deadline\left(\text{first $\kappa$ bytes of}\left(\left(\append_{i<j<2\cdot\numberofscoops}h_j\right)\append\seed\right)\right).
    \]
  \item\label{step:nonce_construction:final_hash}
    Let
    \[
    \finalhash=h_\deadline\left(\left(\append_{0\le j<2\cdot\numberofscoops}h_j\right)\append\seed\right).
    \]
  \item For each $i$ from $0$ to $2\cdot\numberofscoops-1$, reassign
    $h_i$ to $h_i\xor\finalhash$.
  \item\label{step:nonce_construction:swap}
    For each odd $i$ from $1$ to $2\cdot\numberofscoops-1$, swap
    $h_i$ with $h_{2\cdot\numberofscoops-i}$.
  \end{enumerate}
\end{alg}
%
As usual in this context, the complexity of the algorithm is better expressed
in terms of how much data is hashed, since that is the expensive operation that dominates
Alg.~\ref{alg:nonce_construction}. The following result assumes that
integers use a finite machine representation (for instance, $8$ bytes for \texttt{long} in Java)
and that prologs have a fixed structure and length.
%
\begin{proposition}\label{prop:nonce_construction_complexity}
  The execution of Alg.~\ref{alg:nonce_construction}, in terms
  of $\kappa$ and $\numberofscoops$, requires to hash
  $O(\kappa\cdot\numberofscoops)$ bytes with $h_\deadline$.
\end{proposition}
\begin{proof}
  Step~\ref{step:nonce_construction:first_hash} of Alg.~\ref{alg:nonce_construction}
  hashes at most $\kappa$ bytes and gets iterated $2\cdot\numberofscoops$ times.
  Step~\ref{step:nonce_construction:final_hash} hashes
  $2\cdot\numberofscoops\cdot\size(h_\deadline)+|\pi|+\log(p)$ bytes. Therefore,
  Alg.~\ref{alg:nonce_construction} hashes a number of bytes that is
  \begin{align*}
    &O(2\cdot\kappa\cdot\numberofscoops+2\cdot\numberofscoops\cdot\size(h_\deadline)+|\pi|+\log(p))\\
    \text{(constants)}&=O(2\cdot\kappa\cdot\numberofscoops+2\cdot\numberofscoops)\\
    \text{(since $\kappa>0$)}&=O(2\cdot\kappa\cdot\numberofscoops)\\
    &=O(\kappa\cdot\numberofscoops).
  \end{align*}
  \qed
\end{proof}
%
A \emph{plot} is a set of nonces constructed with Alg.~\ref{alg:nonce_construction},
for a finite non-empty set of progressive numbers $P$ and
for a given prolog $\pi$, recorded in the plot.
%
\begin{definition}[Plot]\label{def:plot}
  The set of \emph{plots} is defined as
  \[
  \Plots=\left\{\left<\pi,\nonces\right>\left|\begin{array}{l}
  \pi\in\Prologs,\ \varnothing\not=P\subset\mathbb{N}\text{ is finite}\\
  \text{and }\nonces=\left\{\nonce(p,\pi)\left|\;p\in P\right.\right\}
  \end{array}\right.\right\}.
  \]
\end{definition}
%
The computations of $\nonce(p,\pi)$ and $\nonce(p',\pi)$, for $p\not=p'$,
are completely independent. Therefore, Def.~\ref{def:plot}
implies that the construction of a plot can be optimized on
multicore hardware.

\section{Challenges and Deadlines}\label{sec:challenges_and_deadlines}

\begin{definition}[Challenge]\label{def:challenge}
  The set of \emph{challenges} is
  \[
  \Challenges=\left\{\langle\scoopnumber,\data\rangle\left|
  \begin{array}{l}
    0\le\scoopnumber<\numberofscoops\\
    \text{and }\data\in\mathit{byte}^*
  \end{array}
  \right.\right\}.
  \]
\end{definition}

\section{Blocks and Challenge Generation}\label{sec:challenge_generation}
%
The blocks of the blockchain contain information used for the proof
of space, called \emph{trunk} by borrowing this terminology from~\cite{CohenP19},
other information such as the previous block hash,
and extra information that is irrelevant here, such as a list of transactions.
The latter is not formalized below, since it is not used for the proof of space.
Blocks can be genesis and non-genesis. Both contain their time of creation.
Genesis blocks have nor trunk nor parent and their height is implicitly $0$.
Challenges $c$ are generated in sequence: there is an initial constant challenge for the genesis
blocks, while the subsequent challenges are generated from the trunk of
each non-genesis block $b$ of the blockchain.
In particular, $c$ is \emph{not} computed from the transactions in $b$,
in order to avoid block-grinding attacks (Sec.~\ref{sec:related_work}).
A deadline that answers $c$ is recorded in the trunk of the sons of $b$.
%
\begin{definition}[Trunk, Block]\label{def:trunk}
  The sets of \emph{trunks} and \emph{blocks} are
  \[
  \Trunks=\left\{\langle\height,\deadline\rangle\left|\;\height\in\mathbb{N}\text{ and }\deadline\in\Deadlines\right.\right\}
  \]
  \[
  \GenesisBlocks=\left\{\langle\tau\rangle\mid\tau\in\mathbb{N}\right\}
  \]
  \[
  \NonGenesisBlocks=\left\{\left\langle\begin{array}{c}
  \trunk,\tau,\alpha,\\
  \power,\weightedbeat,\\
  \previousblockhash
  \end{array}\right\rangle\left|\begin{array}{l}
  \trunk\in\Trunks,\\
  \tau\in\mathbb{N},\\
  \alpha\in\mathbb{N},\ \alpha>0,\\
  \power\in\mathbb{N},\\
  \weightedbeat\in\mathbb{N},\\
  \previousblockhash\text{ is a}\\
  \qquad\text{hash for $h_\block$}
  \end{array}\right.\right\}
  \]
  \[
  \Blocks=\GenesisBlocks\cup\NonGenesisBlocks.
  \]
\end{definition}
%
If $b$ is a block, then $b.\tau$ is its creation time (milliseconds from the Unix epoch).
If $b$ is a non-genesis block, then
$b.\alpha$ is called\footnote{In~\cite{SignumPlotting} the term
\emph{base target} is used for it, but we think that \emph{acceleration} is clearer.}
\emph{acceleration} and is used to modulate the time needed to wait for a deadline.
This acceleration changes dynamically from block to block, to cope with
the fluctuations of the mining power in the network. It is the inverse of Bitcoin's difficulty.
The value $b.\power$ expresses how much space has been used to build the path that leads to $b$,
starting from a genesis block; it will be used to select the \emph{best chain} for mining
$b$'s next block. The value of $b.\weightedbeat$ is the average of the interval creation time
for the path that leads to $b$, giving more weight to the last blocks; it will be
compared to $\beat$ (Tab.~\ref{tab:notations}) to understand if the acceleration
must be increased or decreased.
The value of $b.\previousblockhash$ is the hash of the previous block in the path leading to $b$.

The following algorithm shows how the first challenge, for genesis blocks,
its defined. It is a constant that only depends on contextual values (Table~\ref{tab:notations}).
%
\begin{alg}[$\initialchallenge$]\label{alg:initial_challenge}
  The \emph{initial challenge} is the constant
  %
  \begin{multline*}
    \initialchallenge=\\
    \langle\betonat(h_\generation(\sigma_\genesis\append\nattobe(1)))\text{ mod }\numberofscoops,\sigma_\genesis\rangle,
  \end{multline*}
  %
  where $\sigma_\genesis$ is a constant generation signature used for the genesis of the blockchain
  (see Table~\ref{tab:notations}).
\end{alg}
%
The following algorithm shows how a challenge is derived from the trunk of a non-genesis block.
%
\begin{alg}[$\nextchallenge(\trunk)$]\label{alg:next_challenge_from_trunk}
  Let $\trunk\in\Trunks$. The \emph{next challenge for $\trunk$} is
  \begin{multline*}
    \nextchallenge(\trunk)=\\
    \langle\betonat(h_\generation(\sigma\append\nattobe(\trunk.\height+1)))\text{ mod }\numberofscoops,\sigma\rangle
  \end{multline*}
  where
  \[
  \sigma=h_\generation(\trunk.\deadline.\challenge.\sigma\append\trunk.\deadline.\pi).  
  \]
\end{alg}
%
In the following, it will be handy to determine the next challenge for a block. Note that
the next definition only uses the trunk inside the block.
%
\begin{definition}\label{def:next_challenge_from_block}
  Let $b\in\Blocks$. Its \emph{next challenge} is
  \[
  \nextchallenge(b)=\begin{cases}
  \initialchallenge & \text{if $b\in\GenesisBlocks$}\\
  \nextchallenge(b.\trunk) & \text{if $b\in\NonGenesisBlocks$.}
  \end{cases}
  \]
\end{definition}

The definition of the generation signature $\sigma$ for the next challenge,
in Alg.~\ref{alg:next_challenge_from_trunk},
has puzzled us for some time, since~\cite{SignumPlotting} appends a \emph{previous block generator}
to the previous block's generation signature $\trunk.\deadline.\challenge.\sigma$.
That concept, however, is defined nowhere.
We had to dive in the source code of the Signum node
to understand that it is actually
an identifier (more concretely, the public key)
of the creator of the deadline for the previous block
(see \url{https://github.com/signum-network/signum-node/blob/main/src/brs/GeneratorImpl.java}, constructor of \<GeneratorStateImpl>).
We abstract that information through the prolog of the previous deadline, hence this is why
Alg.~\ref{alg:next_challenge_from_trunk} appends
$\trunk.\deadline.\pi$ to define $\sigma$.
This gives us the opportunity to discuss the observation in~\cite{ParkPAFG15}, page~28,
where it is stated that ``this previous block generator seems possible to be grinded, by trying
different sets of transactions to include in a block''. This is false, although it was
hard to grasp without the formalization in
Alg.~\ref{alg:next_challenge_from_trunk}, where it is clear that
$\trunk.\deadline.\pi$ comes from the trunk only, which makes
such grinding attack impossible.

\section{Blockchain Construction}\label{sec:blockchain_construction}

A blockchain is a set of blocks, linked through their $\previousblockhash$ field.
It must contain exactly one genesis block; it has no hash collisions among its blocks;
and all its blocks must satisfy the \emph{consensus rules}.
%
\begin{definition}[Blockchain, Consensus]\label{def:blockchain}
  A \emph{blockchain} is a set $B\subset\Blocks$ such that:
  \begin{enumerate}
  \item there is exactly one $b\in B\cap\GenesisBlocks$, written as $\genesis(B)$;
  \item for each hash $h$ of $h_\block$, there is at most
    one $b\in B$ such that $h_\block(b)=h$, written as $\block(B,h)$;
  \item for each $b\in B$, the predicate $\consensus(B,b)$ holds, where
    %
    \begin{itemize}
    \item if $b\in\GenesisBlocks$, then the only requirement for consensus is that
      $b$ is not created in the future:
      \[
      \consensus(B,b)=b.\tau\ge\tau_\now;
      \]
    \item if $b\in\NonGenesisBlocks$, then $\consensus(B,b)$ is the logical conjunction
      of all the following consensus rules:
      \begin{itemize}
      \item $b$ is not created in the future:
        \[
        b.\tau\ge\tau_\now;
        \]
      \item the deadline of $b$ (that is, $b.\trunk.\delta$) is valid (Def.~\ref{def:deadline_validity});
      \item there are no dangling pointers:
        \[
        p=\block(B,b.\previousblockhash)\text{ exists;}
        \]
      \item the deadline of $b$ answers the challenge of $p$ (Def.~\ref{def:next_challenge_from_block}):
        \[
        \nextchallenge(p)=b.\trunk.\delta.\challenge;
        \]
      \item $b$ is the next block of $p$ \wrt the deadline of $b$ (Def.~\ref{def:next_block}):
        \[
        b=\nextblock(p,b.\trunk.\delta).
        \]
      \end{itemize}
    \end{itemize}
    %
  \end{enumerate}
\end{definition}
%
Note that the above consensus rules, reconstructed and interpolated
from~\cite{SignumPlotting,SignumSource},
do not constrain the prolog of the deadlines in any way:
each block can have an arbitrary prolog. Later, it will be shown why it is useful to
restrain prologs with an extra consensus rule.

\begin{definition}[Blockchain network]\label{def:blockchain_network}
  A \emph{blockchain network} is a
  network of \emph{peers} (computers), each connected to the other peers,
  each holding its own version of a blockchain, for the same genesis block,
  not created in the future.
  Each peer holds a plot (Def.~\ref{def:plot}) in its memory.
  Each peer starts with a blockchain holding only the genesis block and runs two
  algorithms, concurrently: the \emph{block mining} algorithm
  and the \emph{block mined} algorithm.
\end{definition}
%
Note that Def.~\ref{def:blockchain_network} simplifies the picture very much:
the peers are fully connected, never disconnect and never need to synchronize.
Moreover, in practice, peers do not hold plots but rather rely on (one or more)
external services (miners) that hold one or more plots.
The goal here is to keep the picture as simple as possible and concentrate on the properties of the
proof of space algorithm only. Def.~\ref{def:blockchain_network} does not pretend
to describe a real blockchain implementation.

The block mining algorithm looks for the most powerful block in blockchain,
mines a new block on top of it, adds it to the blockchain and whispers it to the other peers.
%
\begin{alg}[Block mining]\label{alg:block_mining}
  The \emph{block mining} algorithm of a peer $P$, holding blockchain $B$,
  is the following infinite loop:
  %
  \begin{enumerate}
  \item identify a most powerful\footnote{In theory, more blocks might be the most powerful in blockchain, although this is highly unlikely; in that case, any of them will be chosen.} block $b$ in $B$;
  \item compute $c=\nextchallenge(b)$ (Def.~\ref{def:next_challenge_from_block});
  \item compute $\delta'=\delta(\plot,c)$ (Alg.~\ref{alg:deadline_from_plot}), where $\plot$
    if the plot of $P$;
  \item compute $b'=\nextblock(b,\delta')$ (Def.~\ref{def:next_block});
  \item\label{step:block_mining:wait} wait until $b'.\tau\ge\tau_\now$;
  \item\label{step:block_mining:add} add $b'$ to $B$;
  \item whisper $b'$ to the peers connected to $P$;
  \item go back to step~1.
  \end{enumerate}
\end{alg}
%
The block mined algorithm receives a block whispered from a peer, checks its validity
and adds it to the blockchain.
%
\begin{alg}[Block mined]\label{alg:block_mined}
  The \emph{block mined} algorithm of a peer $P$, holding blockchain $B$,
  is the following infinite loop:
  %
  \begin{enumerate}
  \item wait for a block $b$ whispered from a connected peer $P'$;
  \item\label{step:block_mined:add} if $B\cup\{b\}$ is a blockchain, add $b$ to $B$;
  \item go back to step~1.
  \end{enumerate}
\end{alg}
%
In practice, step~\ref{step:block_mined:add} of Alg.~\ref{alg:block_mined} should
allow the addition only of blocks $b$ that are \emph{powerful enough} to look useful,
in order to avoid filling the memory with useless blocks. This is not relevant in this paper.
Moreover, if the whispered block $b$
at step~\ref{step:block_mined:add} of Alg.~\ref{alg:block_mined} is more
powerful than $b'$ at step~\ref{step:block_mining:wait} of Alg.~\ref{alg:block_mining},
a rational peer would interrupt the wait at step~\ref{step:block_mining:wait}
of Alg.~\ref{alg:block_mining}, discard $b'$ and restart Alg.~\ref{alg:block_mining}
from step~1, since the whispered $b'$ is better than the block $b$ that it is trying to mine,
hence it is wiser to stop wasting time and start mining on top of the new best block $b'$.
Such optimizations are not considered here.

Note that peers check the validity of blocks coming from outside
(step~\ref{step:block_mined:add} of Alg.~\ref{alg:block_mined}), since they do not trust
their connected peers. Instead, they do not check the validity of the blocks that they
mine themselves (step~\ref{step:block_mining:add} of Alg.~\ref{alg:block_mining}),
since they are valid by construction, as shown below.
%
\begin{proposition}\label{prop:mining_is_sound}
  At step~\ref{step:block_mining:add} of Alg.~\ref{alg:block_mining},
  the set $B\cup\{b'\}$ is a blockchain.
\end{proposition}

%\section{Conclusion}\label{sec:conclusion}
%
Our formalization of Signum's consensus is valuable because it sheds light
on a blockchain network that runs since ten years but was missing any formal definition.
Moreover, it allowed us to understand that Signum is free from block grinding attacks
and is largely protected from challenge grinding attacks.

In the context of proof of space consensus, Signum's advantage is its simplicity,
the small size of its proofs (deadlines, around $200$ bytes)
and the absence of an initialization phase and transactions.
However, its main drawback is that it is
perfectly possible to mine new blocks in a proof of work style: instead of storing a plot,
Def.~\ref{def:deadline_from_plot} could recompute the nonces on the fly. This is currently
not convenient, since Alg.~\ref{alg:nonce_construction} is relatively slow, in particular
by using the shabal256 hashing algorithm for $h_\deadline$ (Tab.~\ref{tab:notations}).
But the situation might change in the future
with the use of ASICs. At the end, it will be the relative
increase of ASICs speed and memory size that will decide if Signum remains a
proof of space network or if it becomes more rational to mine it with proof of work.


%%%%%%%%%%%%%%%%%%%%%%%%%%%%%%%%%%%%%%%%%%%%
\bibliographystyle{plain}
\bibliography{biblio}

\end{document}
