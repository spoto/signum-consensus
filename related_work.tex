\section{Related Work}\label{sec:related_work}

\subsection*{Proof of Stake}

In order to overcome the issue with energy consumption, the recent trend in blockchain is to
replace the proof of work with a proof of stake. This comes in many different flavors, but
the shared idea is that mining is limited to a (static or dynamic, exclusive or delegatable)
set of peers, that commit
some cryptocurrency (a \emph{stake}) to gain the right of mining
in turn, or according to some alternation protocol. In general, these can be seen
as Byzantine consensus algorithms, as the one pioneering by Tendermint~\cite{Tendermint14}.
Proof of stake is often criticized for being more centralized and less democratic than proof of work
(\emph{rich becomes richer}).
Moreover, it suffers from what we call the \emph{start-up issue}: as long as the cryptocurrency
of a newborn blockchain has still no value, it is difficult to convince miners to work and
be updated, since there is no incentive in doing so, initially. Starting and maintaining
a newborn blockchain becomes a difficult social and organizational problem. Finally, peers of
a proof of stake blockchain get punished (\emph{slashed}) if they are misbehaving or offline.
This is problematic if, for instance, a peer is offline but has no fault for that: it might be
because of a network connectivity issue or a black-out.

\subsection*{Proof of Space}

Among the alternatives to proof of work and proof of stake, we focus here on proof of space.
In a proof of space blockchain, peers have the right to mine new blocks (and be renumerated for that) if they dedicate
a large chunk of disk memory for mining. The energy consumption of proof of space is almost zero
and no dedicated hardware can be used for mining, currently. Therefore, mining becomes cheap and
more democratic than with a proof of work
Moreover, proof of space allows one to capitalize on unused memory, for free,
while proof of space has always an electricity cost.
The theoretical background of proof of space has been developed, independently,
in the two seminal papers~\cite{AtenieseBFG14} and~\cite{DziembowskiFKP15},
that feature similarities but also significant differences. Both are based
on directed acyclic graphs (DAGs) of high pebbling complexity.
Pebbling, here, is a directed decoration of the nodes of the DAG with hashes, as in
a Merkle tree.
A prover must keep such a (big) DAG and its pebbling on disk, in order to answer, efficiently,
\emph{challenges} proposed by a verifier, with compact proofs that should convince the verifier that
the prover is actually keeping the DAG on disk. These proofs are used for mining new blocks,
instead of the nonce used in the proof of work. A notion of quality is defined for
the proofs, in such a way that the probability of deriving a proof of high quality increases
with the size of the DAG, which is an incentive to dedicating more space for mining.
While~\cite{DziembowskiFKP15} requires space to remain allocated between challenges,
and is consequently called a proof of \emph{persistent} space, \cite{AtenieseBFG14} requires
to allocate space only when anwering challenges and is consequently called
a proof of \emph{transient} space (or a proof of secure erasure, as~\cite{DziembowskiFKP15} calls it).
Both solutions have an initialization phase, when the verifier performs a deeper challenge
of the prover and stores the resulting proof in blockchain, followed by the execution phase,
when the verifier challenges the prover and creates new blocks including the resulting proofs.

In the same line of~\cite{AtenieseBFG14,DziembowskiFKP15}, also~\cite{RenD16} uses
pebbling for stacked expander graphs, to get simpler, more efficient and
provably space-hard solutions.
It works for both proof of transient space and proof of persistent space.
It includes a nice review of previous proof of space
and related techniques: memory-hard functions, proof of secure erasure, provable data possession,
proof of retrievability.

In~\cite{Reyzin23}, the authors study what happens if the prover stores only a part of the file.
Namely, in that case, it should be difficult for the prover to recover the
missing portion of the file, when answering challenges from the verifier.
In the context of graph pebbling, the initialization phase prevents most cheating
(keeping incomplete file or pebbles). They relate the size
of the portion of the file (or pebbles) not kept by the prover to the consequent time complexity degradation
for computing the missing pebbles, later, during the execution phase.
Ideally, the full file and pebbles must be kept in memory
for having no time complexity explosion, but they show that this is not the case for existing solutions.
They provide sufficient conditions for the initialization phase that guarantee the ideal situation.

As shown above, the use of graph pebbling seems to dominate the literature on proof of space.
However, that approach has its drawbacks.
Namely, as stated in~\cite{AbusalahACKPR17}, the proof to include in the blocks is relatively large:
kilobytes but even megabytes for the initial proof, created in the initialization phase for each new
prover (miner) that joins the blockchain. That initialization phase is a problem by itself: it complicates
the protocol and requires to spend cryptocurrency even \emph{before}
starting mining. Compare this with Bitcoin, for instance, that allows one to start mining
and \emph{later} collect cryptocurrency on the way. This is a barrier to the arrival of new
miners, that is, a limitation of democracy.
This is why~\cite{AbusalahACKPR17} proposes an alternative technology, which is the
theoretical background of the Chia network~\cite{Chia}:
a proof of sequential work on top of a proof of space, based on challenges
about the inversion of a random function, for which it has been solved
the well-known issue of time-memory trade-offs. This is not, however, a pure proof of space.
Another alternative is the proof of retrievability in~\cite{JuelsK07}: a large file
is initially sent from the verifier to the prover and later the verifier
challenges, repeatedly, the prover to see if it still keeps the file in storage.
Its apparent simplicity
is jeopardized by the fact that big files must be shared among the provers (miners) and
the verifiers. In the case of a blockchain, they must be shared among \emph{all} (present and future)
peers, for \emph{all} (past, present and future) miners, or otherwise peers could not verify the blocks.
This is impractical, because of the size of the files and because it complicates the protocol.

%Cite~\cite{DziembowskiFKP15}. This seems to be the first description of proofs of space.
%Their algorithm is based on graph pebbling, where a vertex can be pebbled only if the
%its in-going vertices have been pebbled as well. This way they prove a lower bound on the
%complexity of their algorithm. They prove that that lower bound is valid also if a prover
%wants to use its CPU. They prove that the size of the space used by their algorithm
%is a lower bound to the execution cost of the algorithm if no space is reserved.
%Therefore, using proofs of work in a network of proofs of space nodes would
%be computationally too expensive.
%This is sometimes called proof of persistent space.
%There is an initialization protocol for each new prover, that is missing in Burstcoin.

%Cite~\cite{AtenieseBFG14}. Based on DAGs with high pebbling complexity. There are clear similarities
%with~\cite{DziembowskiFKP15}. They actually cite and compare with each other.
%According to~\cite{DziembowskiFKP15}, this article defines a proof of secure erasure,
%that however they call a proof of space. According to~\cite{DziembowskiFKP15},
%their proof of secure erasure
%implies a proof of space but not the other way round. There does not seem to exist
%any implementation. The issue with the size of the proofs would be identical to that
%of~\cite{DziembowskiFKP15}, because of the use of pebbling graphs.
%This is sometimes called proof of transient space: the puzzle function requires lot of
%memory space to compute, but after computation that space con be freed.

\subsection*{Implementations of Proof of Space}

We are aware of only one implementation of graph-pebbling proof of space:
SpaceMint~\cite{ParkKFGAP18}, previously Spacecoin.
It must be stated that SpaceMint is not a distributed blockchain,
but only a prototype implementation of the proof of space protocol of~\cite{DziembowskiFKP15}.
Its code~\cite{SpaceMintCode} has not been maintained in the last nine years.
Nevertheless, this work is very interesting because it states some problems (and solutions) with proof of space
that are related to its actual game-theory and implementation.
For instance, it uses the public key of the verifier as a parameter for the creation
of the graph pebbling, to combat the creation of mining pools, often seen
negatively~\cite{MillerKKS15}.
Furthermore, it observes that proof of space, being computationally cheap, introduces
specific \emph{nothing-at-stake} problems, the most relevant being that:
%
\begin{enumerate}
\item miners might find it profitable to mine multiple chains simultaneously (not only the best chain),
  hoping to find a better chain with a history change;
\item miners might find it profitable to mine
  (and whisper to the peers) many alternative new blocks, each holding
  different transactions, hoping to find better solutions for the future challenges (\emph{block grinding}).
\end{enumerate}
%
In both cases, miners will end up mining with a mixture of proof of space and proof of work,
which nullifies the benefits of proof of space.
In~\cite{ParkKFGAP18}, a solution to the first problem is suggested, where
previous blocks are forced to affect future blocks only in a limited way
(the next challenge is derived from the hash of a block deep in the past)
and where peers get punished if they
whisper more children blocks of the same father, through a penalty transaction that
includes the children blocks, for evidence.
The solution suggested for the second problem, instead, consists in making the challenges
(and consequently the proofs) independent from the content of the transactions, by splitting the
blockchain in a proofs blockchain and in a transactions blockchain: only the first is used for mining,
and the two are bound through a signature with the private key of the miner.
As a result, the next challenges (and proofs) depend on previous blocks from the proofs blockchain only.
Moreover, this work includes actual experiments, that provide an estimation of the size of the proofs.
Namely, those used in the initialization phase are between two and three megabytes, while those used
in the execution phase can be optimized to around 100 kilobytes. Such proofs must be persistently stored in
blockchain (for each new miner, in the first case, and for each new block, in the second).
This makes the size of the blockchain much larger than in the case of Bitcoin, and requires
that a miner holds cryptocurrency to store proofs even before starting mining.
We observe that also the use of penalty transactions, internally including blocks, is problematic, since
transactions are, in turn, included in blocks, which makes it impossible to define
a maximal block size, a basic security requirement of every blockchain.

%Cite~\cite{DworkN92}. The origin of proof of work. Mail senders must compute some
%work to have their email accepted by the recipient. This work includes
%the address of the recipeint and the date of sending, in order to avoid
%work recycling. Typically, it consists in adding extra data at the end of the email
%to that its hash is smaller than a predefined constant.

Signum~\cite{Signum}, previously Burstcoin, is another implementation of proof of space.
It is not based on graph pebbling: in its proof of space algorithm, that they
call proof of capacity, each miner allocates a large file of hashes, called a
\emph{plot} file. Peers that want to mine new blocks contact miner(s) to ask for a
\emph{deadline}\ie for a compact data structure with a quality measure (its waiting time) that
is inversely proportional to the size of the plot file used by the miner(s). Such deadline is
inserted in the new mined blocks as its proof of space. The simplicity of the protocol
follows from the fact that there is no initialization phase: each miner creates its plot file
independently and off-line: it is not shared and no initial big proof is stored in
blockchain, as it was the case, instead, for SpaceMint. Moreover, Signum is a full-fledged blockchain,
started in 2014, not just a prototype implementation of its protocol. It includes a smart
contract language, called SmartJ. However, Signum has received very little scientific
attention up to now. There is not even a formalization of its proof of space protocol,
but only an informal description~\cite{SignumPlotting}.
We understand that there has been some communication between the authors of~\cite{ParkKFGAP18}
and the developers of Signum. Namely, \cite{ParkKFGAP18} states that Burstcoin has some
time/memory tradeoffs (``a miner doing a little extra computation can mine at the same
rate as an honest miner, while using just a small fraction (\emph{e.g.}, 10\%) of the space''),
that the developers of Signum assert to have solved. Similarly, the latest version of Signum
seems to have separated data in a proofs blockchain and a transactions blockchain,
following the idea in~\cite{ParkKFGAP18}. However, there is no evidence that this has been done
correctly.
%Examples can be found at \url{https://github.com/signum-network/signum-smartj}.
%While Hotmoka abstracts away all blockchain details, SmartJ requires programmers to
