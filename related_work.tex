\section{Related Work}\label{sec:related_work}

Proof of work was originally meant as protection against email spam~\cite{DworkN92}:
senders must perform some work to have their emails accepted by recipients.
%The input includes
%the address of the recipient and the date of sending, in order to ban
%recycling of work.
%The algorithm adds extra data at the end of the email,
%which corresponds to the nonce used in Bitcoin.
Ethereum started with proof of work~\cite{AntonopoulosW18} and later
moved to proof of stake. The latter can be seen as a
Byzantine consensus algorithm, as pioneered by Tendermint~\cite{Kwon14}.
Most current blockchains use some form of proof of stake nowadays.

The theoretical background of proof of space was independently developed
in two seminal papers~\cite{AtenieseBFG14,DziembowskiFKP15}.
%They feature both similarities and significant differences.
Both are based on directed acyclic graphs (DAGs) of high pebbling complexity.
Pebbling, here, is a directed hash decoration of the nodes of the DAG, as in a Merkle tree.
A prover must hold such a (big) DAG and its pebbling on disk, in order to answer, efficiently,
challenges with proofs that should convince a verifier that
the prover is actually holding data on disk.
While~\cite{DziembowskiFKP15} requires space to remain allocated between challenges
(proof of \emph{persistent} space), \cite{AtenieseBFG14}~requires one
to allocate space only when answering challenges
(proof of \emph{transient} space or \emph{proof of secure erasure}, as~\cite{DziembowskiFKP15} calls it).
Both solutions have an initialization phase, when the verifier performs a deeper challenge
of the prover and stores the resulting (big) proof in blockchain, followed by an execution phase,
when the verifier challenges the prover for each new block. Also~\cite{RenD16} uses
pebbling for stacked expander graphs, to get simpler, more efficient and
provably space-hard solutions.
It works for both proof of transient space and proof of persistent space.
It includes a nice review of previous proof of space
and related techniques: memory-hard functions, proof of secure erasure, provable data possession,
proof of retrievability.

Time/memory tradeoffs are studied in~\cite{Reyzin23}. They occur if
a prover can store only a part of the data on disk, with a less than proportional
degradation of its mining capabilities.
A good proof of space algorithm should make it difficult for a prover to recover the
missing part, when answering challenges.
In the context of graph pebbling, the initialization phase prevents most cheating
(that is, keeping incomplete data on disk). In~\cite{Reyzin23}, the size
of the portion of the file not kept on disk is related to the consequent time
complexity degradation for computing the missing part.
Ideally, the full file and pebbling must be kept in memory
for having no time complexity explosion, but they show
that this is not the case for existing solutions and
provide sufficient conditions for the initialization phase, that guarantee the ideal result.

The use of graph pebbling seems to dominate the literature on proof of space.
Instead, \cite{AbusalahACKPR17}~proposes an alternative technology that is the
theoretical background of the Chia network~\cite{CohenP19,Chia}:
a proof of sequential work on top of a proof of space, based on challenges
about the inversion of a random function, for which time/memory tradeoffs have been solved.
This is not, however, a pure proof of space.
Another alternative is the proof of retrievability
in~\cite{JuelsK07}: the verifier sends, initially,
a large file to the prover (miner) and later challenges, repeatedly, the prover to see
if it still keeps the file in storage. Its apparent simplicity
is jeopardized by the fact that big files must be shared.
In a blockchain network, they must be shared among \emph{all} (present and future)
peers, for \emph{all} (past, present and future) miners, or otherwise peers
could not verify the blocks.

We are aware of only one implementation of graph-pebbling proof of space:
SpaceMint~\cite{ParkKFGAP18}, previously Spacecoin~\cite{ParkPAFG15}.
It is not a blockchain implementation,
but a prototype of the proof of space protocol of~\cite{DziembowskiFKP15}.
Its code~\cite{SpaceMintCode} has not been maintained in the last nine years.
Nevertheless, \cite{ParkKFGAP18}~exposes interesting problems (and solutions),
related to the actual game theory and implementation of proof of space.
For instance, it uses the public key of the verifier as an input parameter
for pebbling, to discourage the creation of mining pools, often seen
negatively~\cite{MillerKKS15}.
Furthermore, it provides solutions for nothing-at-stake problems. Against block grinding,
it makes challenges independent from the transactions in the blocks, by splitting the
blockchain in a proofs blockchain and in a transactions blockchain:
only the first is used for mining,
and the two are connected with the signature of the miner.
Against challenge grinding, it lets past blocks influence the quality of short sequences
of future blocks only. A similar but more drastic solution, in~\cite{CohenP19}, is to
use the same challenge for several consecutive blocks,
since it is unlikely that a challenge will be good for many consecutive blocks.
Against mining on multiple chains, \cite{ParkKFGAP18} proposes
a technique that spots such behavior and
imposes a penalty transaction against the culprit.
%The latter includes pairs of blocks, as evidence, which is
%problematic since transactions are, in turn, included in blocks and this makes it impossible to define
%a maximal block size, a basic security requirement of every blockchain.
%We have discussed this with K.\ Pietrzak who suggested that this could be actually optimized
%by reporting signed hashes only.

Actual experiments are reported in~\cite{ParkKFGAP18}
that provide an estimation of the size of Spacemint's proofs:
in the initialization phase, they are between two and three megabytes;
in the execution phase, they can be optimized to around 100 kilobytes.
Such proofs must be persistently stored in
blockchain (for each new miner, in the first case, and for each new block, in the second).
This makes the size of the blockchain much larger than in Bitcoin, and requires
that miners hold cryptocurrency to store proofs even before starting mining.

Signum~\cite{Signum}, previously Burstcoin, has been launched in 2014, with possibly
the first ever language for smart contracts, and is still active.
Appendix B of~\cite{ParkKFGAP18} reports a formalization of an old version
of Alg.~\ref{alg:nonce_construction}.

Newborn attacks are considered in~\cite{TangZDWLG0L19}. Their solution is
to split the space for mining on many chains, with an
incentive to allocate, for each chain, a space proportional
to the market value of the chain.
