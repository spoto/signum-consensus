\section{Blockchain Construction}\label{sec:blockchain_construction}

A blockchain is a set of blocks, linked through their $\previousblockhash$ field.
It must contain exactly one genesis block; it has no hash collisions among its blocks;
and all its blocks must satisfy the \emph{consensus rules}.
%
\begin{definition}\label{def:blockchain}
  A \emph{blockchain} is a set $B\subset\Blocks$ such that:
  \begin{enumerate}
  \item there is exactly one $b\in B\cap\GenesisBlocks$, written as $\genesis(B)$;
  \item for each hash $h$ of $h_\block$, there is at most
    one $b\in B$ such that $h_\block(b)=h$, written as $\block(B,h)$;
  \item for each $b\in B$, the predicate $\consensus(B,b)$ holds true
    where
    %
    \begin{itemize}
    \item if $b\in\GenesisBlocks$, then $\consensus(B,b)=b.\tau\ge\tau_\now$ [$b$ is not created in the future]
    \item if $b\in\NonGenesisBlocks$, then $\consensus(B,b)$ is the logical conjunction
      of all the following consensus rules:
      \begin{itemize}
      \item $b$ is not created in the future:
        $\consensus(B,b)=b.\tau\ge\tau_\now$;
      \item there are no dangling pointers:
        $p=\block(B,b.\previousblockhash)$ exists;
      \item the deadline of $b$ answers the challenge of $p$:
        \[
        \nextchallenge(p)=b.\trunk.\deadline.\challenge;
        \]
      \item $b$ is created at the expiration of its deadline:
        \[
        b.\tau=p.\tau+\waitingtime(b.\trunk.\deadline,p.\alpha);
        \]
      \end{itemize}
    \end{itemize}
    %
  \end{enumerate}
\end{definition}
