\section{Introduction}\label{sec:introduction}

A blockchain is a list of \emph{blocks} that contain \emph{transactions}.
Each block $b$ \emph{points} to its previous block $p$
by referring to the hash of $p$. Blocks satisfy some \emph{consensus} rules;
for instance, the timestamp
of $b$ must be larger than that of $p$; the size of $b$ is bounded by a maximal block size and so on.
The exact nature of the transactions is not relevant here. In general, they are
requests to update the state of a global abstract machine; this state might be a ledger of
payments (as in Bitcoin~\cite{Nakamoto08,Antonopoulos17}) or a sort of global RAM where data
structures can be allocated and modified (as in Ethereum~\cite{AntonopoulosW18}).
By using hashes as machine-independent
pointers, blockchains can be distributed in a network of peers.
This is desirable since it entails that data is safely duplicated
in each peer and that there is no special peer that determines the transactions history.
However, peers are free to expand the blockchain at will, independently from the other
peers, and the blockchain becomes a tree rather than a list.
A notion of chain quality is used to incentivize peers to append blocks to the highest-quality chain
(the \emph{best} chain).
Therefore, a peer might replace its current best chain with another, even better chain,
in what ia called a \emph{history change}.

As presented above, peers are free to generate new blocks at maximal speed, flooding the network
with new blocks, making difficult the emergence of a best chain and inducing frequent history changes.
This is an efficiency and a security issue: history changes allow
\emph{double spending}, when the same money is moved in the ledger twice, once in the previous history
and once in the updated history. The actual genious of Nakamoto~\cite{Nakamoto08} (largely) solved
this issue with an extra consensus rule, that requires blocks to a have a binary hash
that starts with at least $\delta$ zeros, directly bound to the quality of the chain.
This means that the creation of a new block runs a \emph{proof of work} algorithm that rotates the
many possible values for a block field, called \emph{nonce}, until the hash of the block satisfies the
consensus rule. This makes the creation of new blocks hard
(for large $\delta$), makes it impossible to create blocks at arbitrary speed and introduces a heavy
incentive to expanding the best chain, rather than creating alternative histories, or otherwise a peer
risks spending work (concretely, electricity) for creating blocks that will be discarded
by the other peers. The \emph{proof of work} is a brute-force algorithm,
because of the non-correlation
property of hash functions. A peer that performs it is said to \emph{mine}
a new block. Miners get remunerated for their work whenever
they mine a new block before all other miners.

The proof of work comes at the price of energy consumption:
the electricity used by Bitcoin is said
to be comparable to that of a medium-sized country; moreover, mining is not egalitarian, because
it is worthwhile only in countries where electricity is cheap; furthermore, the proof of work
runs more efficiently in dedicated, relatively expensive hardware (such as ASICs),
which deviates much from the idea of a democratic and open network.
Therefore, the trend in blockchain technilogy is towards a\emph{proof of stake}.
This comes in many different flavors, but
the shared idea is that mining is limited to a (static or dynamic, exclusive or delegatable)
set of peers, that commit some collateral (a \emph{stake}) to gain the right of mining
in turn, or according to some alternation protocol.
Proof of stake has been criticized for being more centralized and less democratic than proof of work
(\emph{rich becomes richer}).
Moreover, it suffers from what we call the \emph{start-up issue}: as long as the cryptocurrency
of a newborn blockchain has still no value, it is difficult to convince miners to work and
be updated, since there is no incentive in doing so, initially. Starting and maintaining
a newborn blockchain becomes a difficult social and organizational problem. Finally, miners in
a proof of stake blockchain get punished (\emph{slashed}) if they misbehave or are offline, which
might be perceived as unjust if it is the consequence of a network connectivity issue or black-out.

An alternative to proof of work and proof of stake is
\emph{proof of space}~\cite{AtenieseBFG14,DziembowskiFKP15}, where
peers gain the right to mine (and
be renumerated for that) if they can prove to have dedicated a large chunk of disk memory for that.
Its energy consumption is almost zero and no special
hardware can be used for mining, currently. Therefore, mining becomes cheap
and more democratic. Moreover, proof of space allows
one to capitalize on unused memory, for free, while proof of space has always an
inherent electricity cost. Most theoretical formalizations of proof of space are
based on challenges against trees of high pebbling complexity. However, no actual blockchain
has ever been built using such theory, and only a prototype and non-maintained
implementation of the protocol exists~\cite{ParkKFGAP18}. This is possibly so because of the
complex protocol, that requires an initialization phase for each new miner that joins the network,
to write a quite large proof (megabytes) in blockchain. The only fully-fledged implementation
of a blockchain based on proof of space is Signum~\cite{Signum}. However, it comes without
any formal definition or statement.
There are hints that the authors of~\cite{ParkKFGAP18} have informed the developers of Signum of
theoretical issues in their work, that the latters assert to have solved.
Moreover, the latest version of Signum asserts to use ideas from~\cite{ParkKFGAP18} to solve
other theoretical issues. But nothing has been proved nor published about it.
